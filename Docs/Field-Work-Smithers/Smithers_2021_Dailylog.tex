\documentclass{article}\usepackage[]{graphicx}\usepackage[]{color}

\usepackage{alltt}
\usepackage{float}
\usepackage{graphicx}
\usepackage{tabularx}
\usepackage{siunitx}
\usepackage{amssymb} % for math symbols
\usepackage{amsmath} % for aligning equations
\usepackage{textcomp}
\usepackage{natbib}
\topmargin -1.5cm        
\oddsidemargin -0.04cm   
\evensidemargin -0.04cm
\textwidth 16.59cm
\textheight 21.94cm 
%\pagestyle{empty} %comment if want page numbers
\parskip 7.2pt
\renewcommand{\baselinestretch}{1.5}
\parindent 0pt
\usepackage{blindtext}
\usepackage[T1]{fontenc}
\usepackage[utf8]{inputenc}

\title{Smithers - 2021 - Dailylog}

\author{Deirdre Loughnan}

\begin{document}

\maketitle
\section{Pre-fieldwork notes}

Looked into measuring canopy density using a densiometer. In emailing with Suzanne Simard and Jean Roach, I was advised that they are not very accurate and that it would be better to use either fisheye photography or to just visually estimate it. The Angert lab does have a densiometer that we can borrow.  

\section{July 12, 2021}

First day in the field. Resampled plot JCHM03 with Lizzie and Jonathan, and resampled most of JCHM01 and all of JCHM04, with issues and changes made noted below. 
\begin{itemize}
\item Tags of the germinate plots were all moved to the upper left-hand corner when facing the transect from A-F and a second flagged tag added in the bottom right corner
\item For percent cover:
\begin{itemize}
\item We did not record categories that were 0\%
\item Each category could not exceed 100\%
\item For plots with fallen logs over the plot, only include the growing area, ie under the log
\end {itemize}
\item Had issues with the soil moisture meter in the moss, it appears to be less accurate but we did not want to disturb the plots
\item Switched the setting of the soil moisture meter from ML2 to ML3
\item MAD = missing and assumed dead
\item Heights of trees were defined as the maximum point to the shoot apical meristem, not to the tip of the longest branch, some heights are quite different from last year.
\item When using the compass, we assumed the red portion of the arrow was North and would orientate the compass according to that. 
\end {itemize}

\subsection{JCHM03}
\begin{itemize}
\item on right side of the trail when walking up
\item Tree 1: NW - most northern, closets to the trail  
\item Tree 2: SW - ALNVIR with only one stem, few leaves
\item Tree 3: SE - ABILAS
\item Tree 4: NE - no tag, ABILAS but dead
\item Transect runs from 2 to 4
\item Subplot E: moved flag to under the log
\item Found ABILAS 33 tag on the ground, retagged the tree 1 m away
\item ABILAS 06-09 found on ground, reattached to the nearest ABILAS
\end{itemize}

\subsection{JCHM01}
\begin{itemize}
\item on left side of trail
\item Tree 1: S - closest to trail
\item Tree 2: NE - attached to ABILAS
\item Tree 3: SE - attached to ABILAS
\item Tree 4: SW - attached to ABILAS, also close to the trail
\item transect runs from 2 to 4
\item Could not find germinate plot B -- found on July 13
\end{itemize}

\subsection{JCHM04}
\begin{itemize}
\item on the right side of the trail
\item Found this plot when trying to find JCHM02
\item Very unclear which tree was which, the shape was of a skewed rhombus but below are my best guess
\item Tree 1:  - very close to the trail but closer than it should be to make a square
\item Tree 2: N - very close to the trail and much farther back then the germinate plots suggest
\item Tree 3: NE/E - attached to a menfer
\item Tree 4: S - attached to ABILAS
\item GPS'd subplot A for simplicity
\item Diagonal goes S-SW
\item Plot B: Had to take soil moisture around the branch of the nurse log because it was too hard
\item Plot D: Log suspended over the plot was included in the percent cover -- decided to not do this moving forward
\end{itemize}

\section{July 13, 2021}
Went for a hike with Lizzie and Jonathan to find germinates on the twin falls trail in the morning. In the afternoon, we sampled plots 5 \& 6 and added the additional tags to plots 3, 1, 4. 
 
\subsection{JCHM05}
\begin{itemize}
\item on left side of trail
\item Tree 1: NE - closest to trail, large hemlock tagged so can't be seen from trail!
\item Tree 2: SE
\item Tree 3: Couldn't find
\item Tree 4: NW
\item ASSUMING tree numbers, transect runs from 2 to 4 in the NW direction
\item Found an additional seedling in plot F with current quadrate placement 
\end{itemize}

\subsection{JCHM06}
\begin{itemize}
\item on right side of trail
\item Tree 1: NW - closest to trail
\item Tree 2 or 3: SE (only three flags found)
\item Tree 4: N/NW 
\item transect runs from 2 to 4 potentially
\item It was difficult to determine the shape of the plot -- an additional GPS point was taken for the start of the transect
\end{itemize}

\section{July 14, 2021}
Sampled plots 14, 10, 02. Left the car at 8:40, arrived at plot 14 at 10 am. Sampled 14,10 and 2 while walking down. Had lab meeting in the afternoon.

\subsection{JCHM02}
\begin{itemize}
\item on left side of trail
\item Tree 1 or 3: NW/W
\item Tree 2: N/NW
\item Tree 3: Not found
\item Tree 4: SE
\item Reflagged one tree with light blue tape, the others are pink
\item Plot D: said bottom right, but toothpick was found outside and no orientation with bottom right made sense. We moved the plot to the top left.
\item Plot F: Moved plots tag about 10 cm down to include the toothpick.
\end{itemize}

\subsection{JCHM10}
\begin{itemize}
\item on right side of trail
\item Tree 1: S/SW
\item Tree 2: E/SE
\item Tree 3: N/NE
\item Tree 4: W/NW
\item Transect runs from 2-4
\item Subplot C: Had a nurse log suspended over it with a toothpick in it
\item Subplot D: Has a dead leaning tree over the plot - be careful when walking around the plot.
\end{itemize}

\subsection{JCHM14}
\begin{itemize}
\item on right side of trail
\item Tree 1: SW
\item Tree 2: SE
\item Tree 3: NE
\item Tree 4: NW
\item Assume the transect runs from 2-4 and that 4 is the one closest to the trail
\end{itemize}

\section{July 15, 2021}
Set up one of the three new plots and met with Jim Pojar about species identification. Subplots were set up to be 4m from tree 2 and every 4m from the middle of the plot from then on. We only established the four corners of the plot and the six subplots. We did not tag and measure the DBH of the adult trees. 

Jim confirmed that our identifications were correct. One year germinates of ABILAS has longer needles than the TSUHET and when you look at them above, ABILAS are offset while the needles of TSUHET line up to be symmetrical. He recommended looking in more detail at the descriptions in the Franklin plant germinate guide he sent Lizzie. 

Individuals that were greater than 50cm were not included in our measurements of seedlings.

\subsection{JCHM15}
\begin{itemize}
\item on right side of trail, established with light blue flagging tape
\item Tree 1: NW
\item Tree 2: NE
\item Tree 3: SE
\item Tree 4: SW
\item Transect runs from 2-4, NE to SW, downslope
\item Plot A was established on the right
\item Soil was very shallow so it was difficult to measure soil mositure
\end{itemize}

\section{July 16, 2021}
Raining! No soil moisture measurements could be taken (supposed to rain for the next week!). We Sampled one plot and established two new ones. Had to stop working because we ran out of flagging tape.
Today we encountered a offshoot branch from an adult tree in a plot and we did not include it because it was clearly not from a seedling
We also encountered lichen for the first time and included this in the vegetation category of our precent cover measurements. The heather at the higher elevation does look similar to a woody seedling, but the "needles" are darker, glossy, and the general shape is different.
 
\subsection{JCHM07}
\begin{itemize}
\item on right side of trail
\item Tree 2: NW of tree 1 (just off trail)
\item Tree 3: Not found
\item Tree 4: S
\item Transect runs from 4 to 2 SE/S to NW/N, but is very off from 2 at the beginning 
\item This plot was very shrubby which made it difficult to locate the previous plots, we laid out a transect, but still could not find them
\item Plot A: large fallen tree brought another very large tree down beside the plot
\item Plot D: couldn't find, but no germinates last year so established a new one
\item Plot F: couldn't find, but no germinates so we made a new one and flagged the shrub with light blue tape to be easier to find
\end{itemize}

\subsection{JCHM17}
\begin{itemize}
\item on left side of trail
\item Tree 1: NE/E -- light blue flagging 
\item Tree 2: N/NW -- orange flagging
\item Tree 3: SW/W -- light blue flagging 
\item Tree 4: S/SW -- light blue flagging 
\item Transect runs from 2-4
\item Started plot A on the left side of the transect
\item Plot C had a buried conifer branch that we did not measure since it was clearly an offshoot of the adult tree
\end{itemize}

\subsection{JCHM16}
\begin{itemize}
\item on left side of trail
\item Tree 1: S, near the trail
\item Tree 2: N of 1
\item Tree 3: NW
\item Tree 4: W, also near the trail
\item Transect runs from NW to SW, downslope from tree 2 to 4 and heading back to the trail
\item Plot A was started on the right of the transect
\item Plot F also had small individuals that we feel might be from the nearby adult, but it was less obvious than before. We did measure them, since it is better to have the data and not use it.
\end{itemize}

\section{July 17, 2021}
It rained buckets! We were soaking wet and the write in the rain paper became too saturated to use, so we entered all the data on one sheet instead of flipping between the wet sheets (to reduce the risk of ripping them). We finished sampling the remaining 5 plots. Two of the plots were on very steep slopes and difficult to traverse without causing significant damage, the soil was loose and the moss easily disturbed. 

We also revisited the two plots at the base of Hudson that were located along the Seymour Lake trail. We found both plots, but it was difficult to find actual tags. For JCMH01 we only found 8 tags, but for JCMH02 we found 23 tags. We definitely did not find all the tags in either plots, so it could be worth revisiting and trying to find a few more.

\subsection{JCHM08}
\begin{itemize}
\item on right side of trail
\item Tree 1: Right off trail
\item Tree 2: S/SE, right off trail too
\item Tree 3: Couldn't find
\item Tree 4: N of trail
\item Transect runs from NW/N to SE 
\item Had difficulty finding the plot A because of the plot orientation
\end{itemize}

\subsection{JCHM13}
\begin{itemize}
\item on right side of trail
\item Tree 1: W of trail, very close
\item Tree 2: NE, start of transect
\item Tree 3: E/SE, short ABILAS
\item Tree 4: Couldn't find
\item Transect runs E to W, ending at large tree near the trail
\end{itemize}

\subsection{JCHM12}
\begin{itemize}
\item on right side of trail, uphill
\item Tree 1: NW
\item Tree 2: NE, visible to the right of tree 1
\item Tree 3: Did not find
\item Tree 4: Did not find
\item Transect runs from E to W from the 2nd tree with no bark
\item Transect starts flat but then goes up sharply
\end{itemize}

\subsection{JCHM11}
\begin{itemize}
\item on right side of trail at the bend
\item Tree 1: S
\item Tree 2: SE
\item Tree 3: N, a very large hemlock uphill
\item Tree 4: NW
\item Transect runs SE to NW
\item The transect starts at 2, but appears pretty close to 3
\item F is up on the steep hill just above the boulder and you can see the trail from it
\end{itemize}

\subsection{JCHM09}
\begin{itemize}
\item on left side of trail
\item Tree 1: E, can see it from the trail, up a slight slope
\item Tree 2: E/NE, to the right of tree 1 if entering the plot
\item Tree 3: Did not find
\item Tree 4: Did not find
\item Transect runs from SE to NW, starting at 2, up a very steep hill
\item Could not find all the corners, but did stumble upon the subplots and was able to figure it out from there
\end{itemize}

\end{document}